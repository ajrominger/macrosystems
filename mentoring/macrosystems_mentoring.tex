\documentclass[11pt]{article}
\usepackage[top=0.9in, left=0.9in, right=0.9in, bottom=0.9in]{geometry}
\geometry{letterpaper}
\usepackage{graphicx}
\usepackage{setspace}
\usepackage{amssymb}
\usepackage{amsmath}
\usepackage{epstopdf}
\usepackage{xcolor}
\usepackage{colortbl}
\usepackage{array}
\usepackage{textcomp}
\usepackage{hyperref}
\usepackage[numbers, sort&compress]{natbib}
\usepackage{fancyhdr}
\pagestyle{fancy}
\fancyhead{}
\fancyhead[LO,LE]{Rominger, {\it et al.}}
\fancyhead[RO,RE]{Postdoctoral Mentoring Plan}

\usepackage[T1]{fontenc}
\usepackage{titling}
\setlength{\droptitle}{-5em}

\usepackage{wrapfig}

\usepackage{multirow}

%% make lists more compact
\usepackage{enumitem}
\setitemize{noitemsep,topsep=0pt,parsep=0pt,partopsep=0pt}
\setenumerate{noitemsep,topsep=0pt,parsep=0pt,partopsep=0pt}

% line numbers
% \usepackage{lineno}
% \linenumbers
% \setlength\linenumbersep{5pt}
% \renewcommand\linenumberfont{\normalfont\tiny\sffamily\color{gray}}

%% make sections, subsections and paragraphs more compact
\makeatletter
\renewcommand\section{\@startsection{section}{1}{\z@}%
                                  {-1.8ex \@plus -1ex \@minus 0.2ex}%
                                  {0.1ex \@plus 0.2ex}%
                                  {\normalfont\Large\bfseries}}
\makeatother

\makeatletter
\renewcommand\subsection{\@startsection{subsection}{1}{\z@}%
                                  {-1.8ex \@plus -1ex \@minus 0.2ex}%
                                  {0.1ex \@plus 0.2ex}%
                                  {\normalfont\large\bfseries}}
\makeatother

\makeatletter
\renewcommand\subsubsection{\@startsection{subsection}{1}{\z@}%
                                  {-1.8ex \@plus -1ex \@minus 0.2ex}%
                                  {0.1ex \@plus 0.2ex}%
                                  {\normalfont\bfseries}}
\makeatother

\makeatletter
\renewcommand{\paragraph}{\@startsection{paragraph}{4}{\z@}
  {1ex \@plus 1ex \@minus .2ex}{-1em}
  {\normalfont\normalsize\it}
}
\makeatother


%% make stuff in mini page look ok
\makeatletter
% \usepackage{parskip}
% \setlength{\parindent}{15pt}
% \setlength{\parskip}{\baselineskip}
\newcommand{\@minipagerestore}{
  \setlength{\parskip}{0pt}
  \setlength{\parindent}{15pt}
}
\makeatother


\title{Postdoctoral Mentoring Plan \vspace{-1.5ex}}

\author{} 

\date{}


\begin{document}
\maketitle
\thispagestyle{fancy} 
\vspace{-4em}

Three postdoctoral fellows will be funded through the project, two at
UC Berkeley, and one at the Lawrence Berkeley Lab. Postdoctoral-level
researchers are appropriate for the activities proposed due to the
need for experience and understanding at the intersection between
ecology, microbiology, genomics and soil science. One of the postdocs,
Andrew Rominger, is PI of the current proposal, allowing him to
develop expertise in leading a project, while also working closely
with collaborators. Henrik Krehenwinkel will be developing his
pioneering research in genomics, while extending to microbial
communities in collaboration with the Lawrence Berkeley National Lab
(LBNL). There will be a new postdoc hired at the LBNL, who will work
closely with both Javier Ceja-Navarro and Eoin Brodie, developing
research on microbial diversity. To date, Gillespie has mentored 19
postdocs, one current. Former postdocs have most recently gained
positions at Wageningen University, Netherlands, East Carolina
University, and the California Academy of Sciences, among
others. Brodie has mentored nine postdocs, three current. His former
postdocs have advanced to careers at Lawrence Berkeley National Lab,
VTT in Finland, USDA, and UC Davis, among others.

Each postdoc will have access to UC Berkeley's institutional
resources, including the Berkeley Postdoctoral Association, which
offers resources on careers, funding, and conferences
(\url{postdoc.berkeley.edu}) and the Career Center Services
(\url{https://career.berkeley.edu/Phds/Postdocs.stm}). A particular
advantage of the proposed set of postdoctoral fellows is that the PIs
provide complementary expertise in macroecological and genomic theory,
microbes to arthropods and plants, and conceptual areas that focus on
the intersection between ecology and evolutionary biology. This would
provide each one with broad interdisciplinary training. All three
postdocs will be involved in the following activities:

\begin{enumerate}
\item Mentoring of the Postdocs: Gillespie and Brodie will both work
  with the postdoc on development of their professional skills,
  beginning with a skills self-assessment using the National
  Postdoctoral Association Competency Checklist. With Gillespie and
  Brodie, the postdocs will discuss the results of this assessment and
  will specifically develop activities to improve skills identified as
  being weaker. Weekly individual meetings will allow us to assess
  respective needs.  At the same time, they will participate with the
  graduate students and other postdocs in field and laboratory
  work. Brodie's group also participates in UC Berkeley/LBNL
  terrestrial ecosystems and biogeochemistry working groups where all
  three postdocs will have opportunities for additional presentations
  and feedback from a broad audience.
\item Guest lectures and mentoring: Postdocs will be encouraged to
  give guest lectures in appropriate undergraduate classes to
  develop teaching expertise. Potential classes include Population and
  Evolutionary Genetics, Molecular Ecology, Ecological Genetics,
  Microbial Genomics and Genetics, General Microbiology, and Microbial
  Diversity. Brodie and Gillespie will help each postdoc prepare
  effective undergraduate lectures, graduate discussions, and provide
  post-contact critiques of classroom performance. The postdocs will
  also have the opportunity to mentor students (undergraduate and
  graduate) in both the Gillespie and Brodie research groups.
\item Outreach activities: To develop a broader toolkit for
  communication, each postdoc will work with the UC Berkeley Natural
  History Museums to develop material for the Understanding Evolution
  (\url{evolution.berkeley.edu}) website, designed for science
  teachers of all grade- and experience-levels. The system, which
  couples elements of evolution and ecology, field and laboratory,
  theoretical and empirical, provides an opportunity to effectively
  convey essential yet complex concepts. The postdocs will also be
  expected to become involved in the LBNL's Open House for connecting
  to the local community (\url{www2.lbl.gov/openhouse}) and especially
  the Science at the Theater (\url{uctv.tv/scienceatthetheater})
  events.
\item National meetings and presentations: Postdocs will present their
  research at national meetings, write manuscripts, and participate in
  manuscript peer reviews. Brodie and Gillespie will work with each
  postdoc to improve their scientific writing and oral communication
  skills.
\end{enumerate}

% Together, these activities will allow the project postdocs, not only
% to develop and expand their research expertise, but also to learn how
% to present material to researchers from diverse disciplines, as well
% as to a broader general audience. The training and experience provided
% here would prepare them well for positions in academia, and would also
% provide the skills necessary for working in a broad range of applied
% biological industries and conservation agencies.


\end{document}



