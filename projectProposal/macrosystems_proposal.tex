\documentclass[11pt]{article}
\usepackage[top=0.9in, left=0.9in, right=0.9in, bottom=0.9in]{geometry}
\geometry{letterpaper}
\usepackage{graphicx}
\usepackage{setspace}
\usepackage{amssymb}
\usepackage{amsmath}
\usepackage{epstopdf}
\usepackage{xcolor}
\usepackage{colortbl}
\usepackage{array}
\usepackage[numbers]{natbib}
\usepackage{fancyhdr}
\pagestyle{fancy}
\fancyhead{}
\fancyhead[LO,LE]{Rominger, {\it et al.}}
\fancyhead[RO,RE]{Project Description}

\usepackage[T1]{fontenc}
\usepackage{titling}
\setlength{\droptitle}{-5em}

\usepackage{wrapfig}

\usepackage{multirow}

%% make lists more compact
\usepackage{enumitem}
\setitemize{noitemsep,topsep=0pt,parsep=0pt,partopsep=0pt}
\setenumerate{noitemsep,topsep=0pt,parsep=0pt,partopsep=0pt}


%% make sections, subsections and paragraphs more compact
\makeatletter
\renewcommand\section{\@startsection{section}{1}{\z@}%
                                  {-1.8ex \@plus -1ex \@minus 0.2ex}%
                                  {0.1ex \@plus 0.2ex}%
                                  {\normalfont\Large\bfseries}}
\makeatother

\makeatletter
\renewcommand\subsection{\@startsection{subsection}{1}{\z@}%
                                  {-1.8ex \@plus -1ex \@minus 0.2ex}%
                                  {0.1ex \@plus 0.2ex}%
                                  {\normalfont\large\bfseries}}
\makeatletter

\makeatletter
\renewcommand\subsubsection{\@startsection{subsection}{1}{\z@}%
                                  {-1.8ex \@plus -1ex \@minus 0.2ex}%
                                  {0.1ex \@plus 0.2ex}%
                                  {\normalfont\bfseries}}
\makeatletter

\makeatletter
\renewcommand{\paragraph}{\@startsection{paragraph}{4}{\z@}
  {1ex \@plus 1ex \@minus .2ex}{-1em}
  {\normalfont\normalsize\it}
}
\makeatother


\title{Combining Gradients of Space and Time to Understand Biodiversity Dynamics in the Hawaiian Islands \vspace{-1.5ex}}
\author{}
\date{}


\begin{document}
\maketitle
\thispagestyle{fancy} 
\vspace{-4em}


\section*{Synopsis}

Some text

\section{Background}

We posit that two primary classes of non-steady state exist and can be better understood by combining comparative population and phylogenetic insights across multiple species and ecological theory. The first class of non-steady state occurs when a biological assemblage is undergoing succession following disturbance or formation of new habitat; in this case populations of most species in the community and species composition itself will be in flux due to the stochasticity of immigration and small population sizes. In such a situation the assemblage may be expected to eventually converge on a steady state. Recovery from disturbance, range expansion following climate change and primary succession are all potential examples of such non-steady state. The second case occurs when novel mechanisms actively drive an assemblage away from steady state; such mechanisms could include escalatory species interactions or rapid diversification and adaptation in the face of newfound selective pressures. In both cases idealized ecological theory should fail to predict the static biodiversity patterns of the system and departures from population genetic theory should indicate what demographic dynamics are associated with the failure of ecological theory. We posit that two primary classes of non-steady state exist and can be better understood by combining comparative population and phylogenetic insights across multiple species and ecological theory. The first class of non-steady state occurs when a biological assemblage is undergoing succession following disturbance or formation of new habitat; in this case populations of most species in the community and species composition itself will be in flux due to the stochasticity of immigration and small population sizes. In such a situation the assemblage may be expected to eventually converge on a steady state. Recovery from disturbance, range expansion following climate change and primary succession are all potential examples of such non-steady state. The second case occurs when novel mechanisms actively drive an assemblage away from steady state; such mechanisms could include escalatory species interactions or rapid diversification and adaptation in the face of newfound selective pressures. In both cases idealized ecological theory should fail to predict the static biodiversity patterns of the system and departures from population genetic theory should indicate what demographic dynamics are associated with the failure of ecological theory. We posit that two primary classes of non-steady state exist and can be better understood by combining comparative population and phylogenetic insights across multiple species and ecological theory. The first class of non-steady state occurs when a biological assemblage is undergoing succession following disturbance or formation of new habitat; in this case populations of most species in the community and species composition itself will be in flux due to the stochasticity of immigration and small population sizes. In such a situation the assemblage may be expected to eventually converge on a steady state. Recovery from disturbance, range expansion following climate change and primary succession are all potential examples of such non-steady state. The second case occurs when novel mechanisms actively drive an assemblage away from steady state; such mechanisms could include escalatory species interactions or rapid diversification and adaptation in the face of newfound selective pressures. In both cases idealized ecological theory should fail to predict the static biodiversity patterns of the system and departures from population genetic theory should indicate what demographic dynamics are associated with the failure of ecological theory. We posit that two primary classes of non-steady state exist and can be better understood by combining comparative population and phylogenetic insights across multiple species and ecological theory. The first class of non-steady state occurs when a biological assemblage is undergoing succession following disturbance or formation of new habitat; in this case populations of most species in the community and species composition itself will be in flux due to the stochasticity of immigration and small population sizes. In such a situation the assemblage may be expected to eventually converge on a steady state. Recovery from disturbance, range expansion following climate change and primary succession are all potential examples of such non-steady state. The second case occurs when novel mechanisms actively drive an assemblage away from steady state; such mechanisms could include escalatory species interactions or rapid diversification and adaptation in the face of newfound selective pressures. In both cases idealized ecological theory should fail to predict the static biodiversity patterns of the system and departures from population genetic theory should indicate what demographic dynamics are associated with the failure of ecological theory. We posit that two primary classes of non-steady state exist and can be better understood by combining comparative population and phylogenetic insights across multiple species and ecological theory. The first class of non-steady state occurs when a biological assemblage is undergoing succession following disturbance or formation of new habitat; in this case populations of most species in the community and species composition itself will be in flux due to the stochasticity of immigration and small population sizes. In such a situation the assemblage may be expected to eventually converge on a steady state. Recovery from disturbance, range expansion following climate change and primary succession are all potential examples of such non-steady state. The second case occurs when novel mechanisms actively drive an assemblage away from steady state; such mechanisms could include escalatory species interactions or rapid diversification and adaptation in the face of newfound selective pressures. In both cases idealized ecological theory should fail to predict the static biodiversity patterns of the system and departures from population genetic theory should indicate what demographic dynamics are associated with the failure of ecological theory. We posit that two primary classes of non-steady state exist and can be better understood by combining comparative population and phylogenetic insights across multiple species and ecological theory. The first class of non-steady state occurs when a biological assemblage is undergoing succession following disturbance or formation of new habitat; in this case populations of most species in the community and species composition itself will be in flux due to the stochasticity of immigration and small population sizes. In such a situation the assemblage may be expected to eventually converge on a steady state. Recovery from disturbance, range expansion following climate change and primary succession are all potential examples of such non-steady state. The second case occurs when novel mechanisms actively drive an assemblage away from steady state; such mechanisms could include escalatory species interactions or rapid diversification and adaptation in the face of newfound selective pressures. In both cases idealized ecological theory should fail to predict the static biodiversity patterns of the system and departures from population genetic theory should indicate what demographic dynamics are associated with the failure of ecological theory.


\begin{wrapfigure}[]{r}{0.4\textwidth}
  \label{fig:scale} 
  \vspace{-30pt}
  \begin{center}
    \includegraphics[width=0.4\textwidth]{../nsfBIO/fig_scale.pdf}
  \end{center}
  \vspace{-20pt}
  \caption{\footnotesize Different data sources access different
    scales of space and time. Integrating these data into synthetic
    analyses allows for hypotheses to be tested across scales but
    requires new statistical approaches to deal with the disparate
    sampling processes underlying each data type.}
  \vspace{-20pt}
\end{wrapfigure}


%% suppresses bibliography, then make tex file to compile .bbl in separate doc
\bibliographystyle{abbrvnat}
\setbox0\vbox{\bibliography{../macrosystems.bib}}


\end{document}



