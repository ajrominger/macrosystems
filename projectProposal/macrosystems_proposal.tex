\documentclass[11pt]{article}
\usepackage[top=0.9in, left=0.9in, right=0.9in, bottom=0.9in]{geometry}
\geometry{letterpaper}
\usepackage{graphicx}
\usepackage{setspace}
\usepackage{amssymb}
\usepackage{amsmath}
\usepackage{epstopdf}
\usepackage{xcolor}
\usepackage{colortbl}
\usepackage{array}
\usepackage[numbers]{natbib}
\usepackage{fancyhdr}
\pagestyle{fancy}
\fancyhead{}
\fancyhead[LO,LE]{Rominger, {\it et al.}}
\fancyhead[RO,RE]{Project Description}

\usepackage[T1]{fontenc}
\usepackage{titling}
\setlength{\droptitle}{-5em}

\usepackage{wrapfig}

\usepackage{multirow}

%% make lists more compact
\usepackage{enumitem}
\setitemize{noitemsep,topsep=0pt,parsep=0pt,partopsep=0pt}
\setenumerate{noitemsep,topsep=0pt,parsep=0pt,partopsep=0pt}


%% make sections, subsections and paragraphs more compact
\makeatletter
\renewcommand\section{\@startsection{section}{1}{\z@}%
                                  {-1.8ex \@plus -1ex \@minus 0.2ex}%
                                  {0.1ex \@plus 0.2ex}%
                                  {\normalfont\Large\bfseries}}
\makeatother

\makeatletter
\renewcommand\subsection{\@startsection{subsection}{1}{\z@}%
                                  {-1.8ex \@plus -1ex \@minus 0.2ex}%
                                  {0.1ex \@plus 0.2ex}%
                                  {\normalfont\large\bfseries}}
\makeatletter

\makeatletter
\renewcommand\subsubsection{\@startsection{subsection}{1}{\z@}%
                                  {-1.8ex \@plus -1ex \@minus 0.2ex}%
                                  {0.1ex \@plus 0.2ex}%
                                  {\normalfont\bfseries}}
\makeatletter

\makeatletter
\renewcommand{\paragraph}{\@startsection{paragraph}{4}{\z@}
  {1ex \@plus 1ex \@minus .2ex}{-1em}
  {\normalfont\normalsize\it}
}
\makeatother


\title{Combining Gradients of Space and Time to Understand Biodiversity Dynamics in the Hawaiian Islands \vspace{-1.5ex}}
\author{}
\date{}


\begin{document}
\maketitle
\thispagestyle{fancy} 
\vspace{-4em}


\section*{Synopsis}


\section{Background}

Earth systems are nearing or already past a global tipping point past which the biological diversity, fundamentally important to the functioning of all natural and human-engineered ecosystems \citep{barnosky2012}. Beyond this phase transition, the processes regulating biodiversity will change, and the dynamics of their resultant biological systems, from clades to ecosystems, will become non-equilibrial \citep{barnosky, boettinger}.  Despite the pressing need to better understand biodiversity dynamics, we still have only a very rudimentary predictive capacity; we must be able to address how has biodiversity been shaped in the past, what are the expectations as we move into the future and how will associated ecosystems adapt to global change? We must be able to tackle these questions from plots to biomes, as by their nature, phase transitions affect all scales of organization. Advances in our understanding of specific ecosystem components is also idiosyncratic.  Remote sensing and distributed biogeochemical monitoring \citep{asner, NEON} rapidly advancing ecosystem modeling, while similar large scale study of organismal processes, from genetics to populations and communities, lags behind, especially for ``dark taxa'' such as arthropods and microbes.

A grand challenge in understanding the origins and future of biodiversity is to disentangle the influence of evolutionary and historical processes operating at larger spatiotemporal scales from ecological processes operating at smaller scales (Lessard et al. 2012).  Feedbacks between processes along this evolutionary-ecological continuum drive non-equilibrial biodiversity dynamics \citep{brown1971, ricklefsNeutral, rosindell, rominger2015}. The consequences of non-equilibrial dynamics are profound for ecosystem function \citep{XXX}, and based on state shifts in the geologic past, the consequence for evolution will persist for millions of years \cite{erwin}.

What makes integrating evolutionary and ecological processes difficult is that while we can observe and test local ecological phenomena, we must usually infer evolutionary processes from current observations, often at larger spatial and temporal scales, thus modeling processes across scales is key. Efforts to reconcile the interaction of ecological and evolutionary processes have not taken this scaling approach.  Instead they largely adopt one of two methodologies, one making maximal use of extensive sets of spatial data for broad comparative studies (Chase and Myers 2011; Belmaker and Jetz 2012; Carstensen et al. 2013), another using detailed phylogenetic hypotheses across entire lineages to provide insights into change over evolutionary time coupled with data on current ecological traits (Wiens et al. 2011; Anacker and Harrison 2012; Graham et al. 2014). Both these approaches lack the ability to predict how evolutionary history influences an ecosystem’s response to environmental change and how ecological time-scale mechanisms such as dispersal and environmental filtering scale up to influence evolutionary outcomes such as adaptation, speciation and extinction.  The propensity for systems to transition into non-equilibrial states cannot be assessed given current means of synthesizing ideas from ecology with those from evolution.  A lack of cross-scale biodiversity data (from plots to landscapes and genes to species) combined with a lack of theoretical framework, limit this synthesis.

In this proposal we seek to develop a solution to the dearth of predictive biodiversity knowledge by testing ecological theory that identifies and predicts non-equilibrial systems in the unique eco-evolutionary natural experiment afforded by the Hawaiian archipelago.  We will use three exemplar functional groups---microbes, arthropods and plants---to understand the dynamics of vastly different ecosystem members, exploring their evolution and large-scale ecological dynamics across environmental gradients to understand responses to future change.  We propose a novel and revolutionary approach to produce massive ecological, population genetic and phylogenetic data on an unprecedented scale. We will do so by joining and building on the effort to advance macrosystems ecology already underway by the National Ecological Observatory Network.  Our project will contribute theoretical constructs for use across NEON sites and bioinformatic tools to advance the rate and dimensionality of biodiversity data gathered at these sites.


\subsection{Theory provides a lens on non-equilibrium processes}

The neutral theory of biodiversity and biogeography \citep[NTB;][]{hubbell2001} brought mechanistically simplified theory to the forefront of ecological research \citep{chave, rosindell}.  Recent advances have continued the approach of simplifying mechanistic assumptions, with the maximum entropy theory of ecology (METE) drawing from the probabilistic properties of large, randomly assembled systems studied in statistical physics \citep{harte2011}.  Any predictive failure of these mechanistically simple theories can be most parsimoniously attributed to their lack of non-equilibrial evolutionary process: NTB assumes slow, gradual evolution which is not supported in real communities \citep{ricklefsNeutral} and both theories ignore unique niches evolved through novel interactions among populations and between populations and their environments \citep{XXX}.

These theories predict ecological assemblages in equilibrium, after the fundamentally non-equilibrium processes of speciation, colonization and extinction have reached a stationary balance.  Not surprisingly, both theories arrive at the same predicted species abundance distribution \citep[Fisher’s log-series, long established as a null or neutral model;][]{fisher}, with NTB producing modifications to the log-series based on dispersal limitation.  We will focus on METE due to this mathematical convergence, and the greater number of testable predictions made by METE \citep{harte2011}.  Deviations from METE allow us to identify ecological systems out of equilibrium \citep{harte2011, rominger2015}.  Drivers of such non-equilibrium include rapid assembly following disturbance \citep{harte2011} and constraints imposed by evolutionary history and non-neutral adaptive differences between species that violate the statistical assumptions underlying the principle of maximum information entropy \citep{rominger2015}.  In order to harness these promising properties of METE as a non-equilibrium diagnostic tool more testing is needed to understand how exactly the ecological and evolutionary setting of a community predicts its deviation from METE.  The chronological age structure of the Hawaiian archipelago offers the perfect natural experiment to achieve this.


\subsection{Hawaii as a eco-evolutionary laboratory}

Remote island archipelagos provide an opportunity to integrate ecological and evolutionary processes, advancing our understanding of the regulation of biodiversity through the lens of theory.  This is particularly true when the component islands are arranged chronologically, as is found in ``hotspot'' islands, with multiple discrete volcanoes each providing elevational gradients with contrasting physiological barriers (temperature and rainfall) across the gradient, and that have incurred radically different anthropogenic impacts. Such islands not only provide simple and discrete systems, but they also offer advantages for ecological and evolutionary study because of their known age, and varying area, allowing them to serve as ``natural laboratories'' (Simon 1987; Chadwick et al. 2007; Gillespie and Clague 2009). The replicate nature of these systems provides a means for teasing apart variables so as to determine how ancient and ongoing processes of colonization, adaptation, speciation, invasion, and extinction have molded biodiversity, while the age chronology allows analysis of communities that are just becoming established, to older communities $\geq$ 5 million years. Thus, the replication of discrete elevational gradients across the chronosequence allows insights into one of the outstanding challenges in biodiversity— how do ecological processes (e.g. colonization, invasion, and ecological fitting) interact to give rise to larger and longer term processes of adaptation and species diversification? Our proposed research aims to use the Hawaiian Island chronosequence as a natural laboratory for understanding community interactions that underlie biodiversity dynamics and environmental change by incorporating new technologies and theoretical approaches, coupled with standardized sampling protocols, thus providing temporal replicates of the same ecological and evolutionary processes across gradients of elevation (Kueffer and Fernandez-Palacios 2010).

Within this natural laboratory we will focus on microbes, arthropods and plants.  In native montane Hawaiian forests these broad functional groups determine the majority of ecosystem function. They also encompass a wide spectrum of ecological and evolutionary strategies, from rapid generation time to slow, vagile to dispersal limited, and allopatric and ecological speciation.  They also form well studied interaction networks driven both by ecological and evolutionary forces, with important consequences for biodiversity maintenance \citep{bascompte}.  There is also a strong foundation of research on all three groups through the research of PIs Rominger, Gillespie, Gruner and Krehenwinkel (Dimensions in Biodiversity grant on arthropods and microbes); PI Chase (plants); PI Brodie (microbes).


\subsection{NEON site integration}

Understanding how environmental change will alter the feedback between ecology and evolution and drive biodiversity out of equilibrium is at the core of our proposal. The NEON site at Puu Makaala Natural Area Reserve on Hawaii Island (which PI Giardina is instrumental in establishing) will provide the core measures needed to quantify the abiotic environment.  We will replicate these measurements across gradients of elevation and precipitation, using ground-truthed remotely sensed measurements to provide both fine grain and broad-scale environmental data products.


\subsection{Next generation sequencing approach to macroscale biodiversity}

The same ability to generate massive amount of environmental data via remote sensing does not exist for organismal ecology and evolution.  As part of our Dimensions in Biodiversity grant, PIs Rominger and Krehenwinkel are developing laboratory and bioinformatic methods to obtain sequence data, and estimates of abundance and biomass for thousands to millions of arthropods collected via ecological sampling.  As part of the current proposal this promising new approach will be developed into an open source lab protocol and software package that can be distributed across all NEON sites.


\section{Proposed Research}
In this section we need to tie the ideas from the background to specific objectives, questions and hypotheses.  Then we need to tie those objectives, etc. to concrete methods.  Basically, if we care about how environmental change and evolutionary history will drive communities out of equilibrium, why are we going to do XYZ?
\subsection{Research Objectives}
Developing open source tools to advance multi-dimensional ecological and evolutionary data generation across scales
Quantify evolutionary and macroecological patterns across environmental and age gradients
Evaluate environmental and evolutionary drivers of assembly
Understand how evolutionary processes and environmental change drive communities out of equilibrium
Scale signals of community dis-equilibrium and its consequences from local to regional



\subsection{Significance and Rationale}




\subsection{Methods}

\subsubsection{Integration with NEON and sampling design across environmental and age gradients}

\paragraph{NEON site.}
The goal of NEON is to provide ecological data at multiple spatial and temporal scales. We plan to use Pu'u Maka'ala Natural Area Reserve on the Big Island of Hawaii. This site is planned as a core terrestrial site. We aim to combine the data that will be collected here, with data collected at other sites across the Hawaiian Islands, in order to understand regional-scale ecological processes and how these respond to change over space and time.


\paragraph{Complementary sites.}
We will collect data in an explicit nested way that allows integration with the NEON-generated data, while using data from the entire terrestrial region of the Hawaiian Islands to provide information on processes across multiple scales. Data will be gathered across elevation and precipitation gradients from evolutionarily old, middle aged and young islands (Kaua’i, Maui, and Hawai’i).  On each island we will establish 6 sites to span precipitation gradients from XXXXmm -- XXXXmm annual rainfall and 900 m -- 2500m elevation.  On Hawai'i Island we will use the area adjacent to the Pu'u Maka'ala NEON site as one of these 6 sites.  Each site will consist of 3 replicate plots to insure thorough coverage of local variation. The sampling locations and design are given in Figure \ref{fig:map}.


\paragraph{Sampling approach.}
We will select sites in clearly defined ohia/koa montane, wet and mesic forest communities. The rationale here is that (i) Ohia (Metrosideros polymorpha) is the dominant canopy tree in these forests, forming a nearly continuous layer, with patches of sub-dominant koa (Acacia koa) and numerous associated understory trees, shrubs, herbs, and ferns. This forest type (and the presence of Metrosideros in particular) has been used as an important landscape feature in our ongoing work through the Hawaii Dimensions of Biodiversity. (ii) The proposed NEON site is characterized by this forest type. And (iii) Metrosideros growth rate, growth form and chemical composition (all detectable by various satellite and airborne spectroscopic techniques (Asner et al. 2006; Asner and Martin 2009; Asner et al. 2011)) reflects the coupled effects of soil age and fertility, which in turn affects the community of organisms in a given forest stand (Crews et al. 1995; Gruner 2007b). Differences in plant traits can affect the structure of an entire food web through a series of direct and indirect effects (Gruner et al. 2005; Bukovinszky et al. 2008).

In each sampling replicate at each site, we will establish a ___XXXXX____ plot. Within these plots we will sample arthropod abundance and richness, plant (trees $>$ 1cm DBH) abundance and richness, and soil microbial diversity using the same methodology employed at NEON sites.  While NON focuses on ground beetles (Carabidae); mosquitoes (family Culicidae); ticks (order Ixodida), we chose to focus on all arthropods because ground beetles constitute a small proportion of native arthropod diversity in Hawaii, and there are no native mosquitos or ticks.  Specific data are as follows:

Soil Sensors and Measurements: NEON collects and produces data on nutrients, such as carbon, nitrogen and phosphorus, which move through the atmosphere, water bodies, soil, microbes, plants and animals to support understanding of nutrient cycling in a variety of ecosystems. Soil observations include sensor measurements within the tower airshed, broadly distributed collections of soil cores used for physical and biogeochemical analyses, and one-time soil characterization. NEON soil sensor measurements directly characterize soil properties, such as soil moisture and temperature, as well as relevant ancillary measurements (soil respiration, biogeochemistry and nutrient concentrations in fine root biomass). Soil sampling within plots across a site will focus on soil microbial communities (abundance and diversity), metagenomes and transcriptomes, as well as biogeochemical analyses including pH, total carbon (C), nitrogen (N), phosphorous (P) and sulfur (S). Questions that are being examined with this data include, how do changes in temperature alter decomposition rates and microbial activity in different ecosystems across a precipitation gradient? Key measurements include biomass, soil carbon and nutrient pools, carbon exchange, litterfall, decomposition, biomass production, and microbial activity.

Terrestrial Organismal Sampling: Arthropods will be sampled across variable environments using two complementary methods: malaise-style interception traps at ground and canopy levels, and vegetation beating of understory plants. Samples will be preserved in molecular-grade ethanol for identification and metabarcoding. Plants will be sampled.....  We will measure leaf chemistry and biomass.

Remote Sensing and Measurements of Gases: The NEON site will track fluxes of gases, such as carbon dioxide (CO2) and water vapor, and collects data about physical and chemical climate conditions, such as temperature, barometric pressure and visible light or Photosynthetically Active Radiation (PAR). Sensors on the NEON tower systems track fluxes of gases (CO2, water vapor) and collects data about physical and chemical climate conditions, such as temperature, humidity, wind, and the amount of gas that is exchanged between the atmosphere and the ecosystem. Towers extend past the top of the vegetation canopy at each site to allow sensors mounted at the top and along the tower to capture the full profile of atmospheric conditions from the top of the vegetation canopy to the ground. Automated tower sensors collect data continuously to capture patterns and cycles across various time periods, ranging from seconds to years. Categories of measurements are physical climate (aerosols, precipitation, radiation, and temperature, pressure and wind; chemical climate (wet deposition, chemistry, isotopes and scalar concentrations); net ecosystem exchange: carbon dioxide (CO2) flux, soil CO flux, water vapor and latent heat flux, sensible heat, total reactive nitrogen (NO2) and ozone (O3)

Airborne Remote Sensing: We will make use of both existing and planned airborne remote sensing data which can provide information on vegetation composition and land cover and will be used in particular to examine the complex mosaic of forest structure and composition.



\subsubsection{Quantifying evolutionary and macroecological patterns using metabarcoding}

Next generation sequencing technology has ushered in a revolution in evolutionary biology and ecology. This revolution has not passed by taxonomy and spurred various new studies in the field of molecular barcoding. The current leap in sequencing throughput allows to routinely perform barcoding studies on bulk samples and analyzing whole ecosystems (Taberlet et al. 2012; Leray & Knowlton 2015; Gibson et al. 2014; Ji et al. 2013). The large scale recovery of species richness, food web structure, cryptic species, identification of juveniles and hidden diversity, e.g. internal parasitoids, promise unprecedented new insights into ecosystem function and assembly (Krehenwinkel et al. 2016; Shokralla et al. 2015; Shokralla et al. 2012; Kress et al. 2015; Kartzinel et al. 2015). While species richness can be routinely identified by sequencing bulk samples, estimating species abundance remains challenging (Elbrecht & Leese, 2015) and severely limits the application of metabarcoding to many studies. We are developing wet lab and bioinformatic methods to overcome this issue and revolutionize the generation of ecological and genetic data. Our pipeline consists of three steps (Fig. \ref{fig:metab}):
Extraction and sequencing of pooled community samples
Matching the resulting sequences to a reference phylogeny for identification
Using Bayesian hierarchical models to reconstruct unbiased estimates of abundance

Step (1) will be released as an open source lab protocol and steps (2-3) will be developed into an open source R package that allows users to implement these methods in their study systems.  We propose that our open source pipeline can be implemented across NEON sites to generate both taxonomic and phylogenetic data for focal taxa.


Figure \ref{fig:metab} showing pipeline for generating and analysing metabarcoding samples.

Preliminary results from controlled experiments show there is a strong correlation between amount of DNA and total number of reads; however, this relationship is variable across taxa (Fig. \ref{fig:metabData}). A Bayesian model is able to capture this variability across taxa (Fig. \ref{fig:metabData}) and thus indicates the success of more general applications of the modeling approach to field collections. 

\paragraph{(1) Extraction and sequencing of pooled community samples.}
We will generate sequence information for mixed arthropod community samples, collected across precipitation gradients on the Hawaiian Archipelago. The samples will be roughly pre-sorted taxonomically and grouped into different body size classes to minimize the confounding factors of abundance and body size in determining amount of DNA per taxon. We will use amplicon sequencing of the COI barcoding region \citep{} which has shown the greatest reliability in preliminary trials.


\paragraph{(2) Matching the resulting sequences to a reference phylogeny for identification.}
In order to resolve the taxonomy of sequences derived from mixed samples we are developing a library of the barcoding region for species across the Hawaiian archipelago, such that unknown sequences can be phylogenetically matched \citep{} to the reference library.  Sequences not found in the tree of all reference sequences will be grafted and their status as a unique operational taxonomic unit assessed using a cuttoff of 3% divergence (Fig. \ref{fig:metab}).  These bioinformatic steps will be included in the R package.

In collaboration with taxonomist and ecologist on Hawaii, we are currently working on generating the barcode reference library for a diverse range of several hundred Hawaiian arthropod taxa. These taxa were sampled across the chronosequence of the Hawaiian Archipelago (Fig. \ref{fig:map}). DNA is extracted from each taxon and reference sequence generated for the mitochondrial COI barcoding region. To achieve a comprehensive sampling of the Hawaiian arthropod diversity, samples from environmental gradients (e.g. precipitation) will be included in this reference collection. Such gradients have been shown to have a profound influence on community composition on Hawaii \citep{zimmermann2012}.

In order to build a robust phylogenetic backbone for our reference library, the genomic DNA extracts for all species will be sequenced using the Illumina HiSeq2500. An assembly of the resulting reads promises to generate near complete mitochondrial genomes and nuclear ribosomal clusters of each taxon. To support the Illumina short read assemblies, we will generate long read information by PacBio sequencing. The resulting sequence information will allow us to reconstruct a well resolved community-phylogenetic framework for ecological hypothesis testing.

\paragraph{(3) Using Bayesian hierarchical models to reconstruct unbiased estimates of abundance}
Bayesian hierarchical models permit inference of key quantities (e.g. abundance) while accounting for multiple sources of error and leveraging heterogeneous data types to facilitate inference \citep{royleDorazio}.  The goal of hierarchically modeling metabarcoding data is to estimate the abundances of species while correcting for known biases inherent in amplicon-based sequencing.  We will account for bias from copy number variation \citep{} and primer affinity \citep{} by directly modeling it, while also using data on the total number of individuals being sequenced, their body sizes, and the phylogenetic relationship between their sequences to constrain the estimates to be more accurate (Fig. \ref{fig:metabMod}).  Furthermore, information from controlled experiments (for example making moc communities of known composition and sequencing those) can be used to constrain prior distributions and obtain even more accurate abundance estimates.

\subsubsection{Modeling evolutionary and environmental drivers of assembly}
\paragraph{How environment and age Maximum entropy theory of ecology}


%% suppresses bibliography, then make tex file to compile .bbl in separate doc
\bibliographystyle{abbrvnat}
\setbox0\vbox{\bibliography{../macrosystems.bib}}


\end{document}



