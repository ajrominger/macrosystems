\documentclass[11pt]{article}
\usepackage[top=0.9in, left=0.9in, right=0.9in, bottom=0.9in]{geometry}
\geometry{letterpaper}
\usepackage{graphicx}
\usepackage{setspace}
\usepackage{amssymb}
\usepackage{amsmath}
\usepackage{epstopdf}
\usepackage{xcolor}
\usepackage{colortbl}
\usepackage{array}
\usepackage{textcomp}
\usepackage{hyperref}
\usepackage[numbers, sort&compress]{natbib}
\usepackage{fancyhdr}
\pagestyle{fancy}
\fancyhead{}
\fancyhead[LO,LE]{Rominger, {\it et al.}}
\fancyhead[RO,RE]{Project Description}

\usepackage[T1]{fontenc}
\usepackage{titling}
\setlength{\droptitle}{-5em}

\usepackage{wrapfig}

\usepackage{multirow}

%% make lists more compact
\usepackage{enumitem}
\setitemize{noitemsep,topsep=0pt,parsep=0pt,partopsep=0pt}
\setenumerate{noitemsep,topsep=0pt,parsep=0pt,partopsep=0pt}


%% make sections, subsections and paragraphs more compact
\makeatletter
\renewcommand\section{\@startsection{section}{1}{\z@}%
                                  {-1.8ex \@plus -1ex \@minus 0.2ex}%
                                  {0.1ex \@plus 0.2ex}%
                                  {\normalfont\Large\bfseries}}
\makeatother

\makeatletter
\renewcommand\subsection{\@startsection{subsection}{1}{\z@}%
                                  {-1.8ex \@plus -1ex \@minus 0.2ex}%
                                  {0.1ex \@plus 0.2ex}%
                                  {\normalfont\large\bfseries}}
\makeatother

\makeatletter
\renewcommand\subsubsection{\@startsection{subsection}{1}{\z@}%
                                  {-1.8ex \@plus -1ex \@minus 0.2ex}%
                                  {0.1ex \@plus 0.2ex}%
                                  {\normalfont\bfseries}}
\makeatother

\makeatletter
\renewcommand{\paragraph}{\@startsection{paragraph}{4}{\z@}
  {1ex \@plus 1ex \@minus .2ex}{-1em}
  {\normalfont\normalsize\it}
}
\makeatother


%% make stuff in mini page look ok
\makeatletter
\usepackage{parskip}
\setlength{\parskip}{\medskipamount}
\newcommand{\@minipagerestore}{
  \setlength{\parskip}{\medskipamount}
  \setlength{\parindent}{15pt}
}
\makeatother


\title{Combining Gradients of Space and Time to Understand Biodiversity Dynamics in the Hawaiian Islands \vspace{-1.5ex}}
\author{}
\date{}


\begin{document}
\maketitle
\thispagestyle{fancy} 
\vspace{-4em}


\section*{Synopsis}


\section{Background}

Biological diversity is nearing or already past a global tipping point
\citep{barnosky2012}. Beyond this phase transition, the processes
regulating biodiversity will change, and the dynamics of their
resultant biological systems, from clades to ecosystems, will become
non-steady state \citep{barnosky2012}.  Despite the pressing
need , our level of understanding of biodiversity dynamics remains
rudimentary .  We must be able to address how biodiversity has been
shaped in the past, what are the expectations as we move into the
future, and how will associated ecosystems respond to global
change. Phase transitions operate across spatial scales and so we must
be able to tackle these questions from plots to biomes in order to
detect and understand non-steady state dynamics. Advances in our
understanding of specific ecosystem components are idiosyncratic.
While remote sensing and distributed biogeochemical monitoring
\citep{asner2012, NEON} are rapidly advancing ecosystem modeling, similar
large scale study of organismal processes, from data generation to
theory development, and from genetics to populations and communities,
lags behind, especially for ``dark taxa'' such as arthropods and
microbes.

Biodiversity results from both evolutionary and historical processes
operating at larger spatiotemporal scales and ecological processes
operating at smaller scales \citep{lessard2012}.  Feedbacks between
processes along this evolutionary-ecological continuum drive
non-steady state biodiversity dynamics \citep{brown1971,
  ricklefs2006neutral, rominger2015}. The consequences of non-steady
state dynamics, based on state shifts in the geologic past, will
persist for millions of years \citep{Erwin1998}. Yet we lack
approaches that synthesize across scales of space and evolutionary
time to understand the consequences of this eco-evolutionary feedback
process.  The propensity for systems to transition into non-steady
states cannot be assessed given current means of synthesizing ideas
from ecology with those from evolution.  Additionally, a lack of
cross-scale biodiversity data (from plots to landscapes and genes to
species) combined with a lack of theoretical framework, limit this
synthesis.


\subsection{Theory provides a lens on non-steady state processes}

Recent theoretical developments have brought mechanistically
simplified theory to the forefront of ecological research
\citep{hubbell2001, rosindell2011TREE, harte2011}.  These simple
theories have been critical because they provide robust null models
against which to compare real biodiveristy patterns in order to
rigorously test the importance of specific mechanisms in shaping
biodiversity.  The maximum entropy theory of ecology
\citep[METE;][]{harte2011} provides one of the most useful null
predictive frameworks because it produces many falsifiable patterns
(the species abundance distribuiton, metabolic rate distribution,
species area relationship and network structure) and is grounded in
the principles of statistical mechanics \citep{harte2011, Jaynes1957}.
METE draws from the probabilistic properties of large, randomly
assembled systems \citep{harte2011} and thus its predictions
constitute a community in statistical steady state.  Statistical
steady state means specifically that a system is governed by only a
few simple state variables, which constitute a state space, and that
no additionally processes limit the system’s ability to freely explore
this state space.  This precise definition is made more clear in Box
1.  Statistical steady state connects to some notions from the
literature on ecological equilibrium, specifically the condition of
stationarity and egodicity \citep{brian1999}, but is in no way tied
\citep{harte2011} to ideas relating to equilibrium as a hypothesized
state that ecosystems may attain or be driven away from
\citep[e.g. not:][]{levin1970, scholz1997, rabosky2009}.

Deviations from METE allow us to identify ecological systems out of
statistical steady state \citep{harte2011, rominger2015}.  Drivers of
such non-equilibrium include rapid assembly following disturbance
\citep{harte2011} and constraints imposed by evolutionary history and
non-neutral adaptive differences between species that violate the
statistical assumptions underlying the principle of maximum
information entropy \citep{rominger2015}.  In order to harness these
promising properties of METE as a non-equilibrium diagnostic tool more
testing is needed to understand how exactly the ecological and
evolutionary setting of a community predicts its deviation from METE.

\begin{wrapfigure}[]{r}{0.6\textwidth}
\colorbox{gray!20}{
  \begin{minipage}{0.6\textwidth}
    \noindent    
    {\bf Box 1: Statistical Steady State}

    Whether or not biodiversity dynamics are governed by stable
    equilibria remains an unsolved question in ecology and evolution
    \citep{rabosky2015amNat, harmon2015amNat}. A
    statistical steady state exists in an ecological community if
    changes in biodiversity occur slowly and in sync with
    environmental changes \citep{harte2011}. The existence (or
    non-existence) of such steady states has wide ranging
    implications. For example, whether conservation should focus on
    conventional preservationalist paradigms or adaptive management
    \citep{levin1999} depends on whether biodiversity is largely in
    statistical steady state or not. Whether biodiversity rapidly and
    consistently tends towards a steady state determines how species
    and the communities they form will respond to global environmental
    change \citep{barnosky2012}.

    We posit that two primary classes of non-steady state exist and
    can be better understood by combining comparative population and
    phylogenetic insights across multiple species and ecological
    theory. The first class of non-steady state occurs when a
    biological assemblage is undergoing succession following
    disturbance or formation of new habitat; in this case populations
    of most species in the community and species composition itself
    will be in flux due to the stochasticity of immigration and small
    population sizes. In such a situation the assemblage may be
    expected to eventually converge on a steady state. Recovery from
    disturbance, range expansion following climate change and primary
    succession are all potential examples of such non-steady
    state. The second case occurs when novel mechanisms actively drive
    an assemblage away from steady state; such mechanisms could
    include escalatory species interactions or rapid diversification
    and adaptation in the face of newfound selective pressures. In
    both cases idealized ecological theory should fail to predict the
    static biodiversity patterns of the system and departures from
    population genetic theory should indicate what demographic
    dynamics are associated with the failure of ecological theory.
  \end{minipage}
}
\end{wrapfigure}

We propose to use islands of the Hawaiian archipeligo to better
understand how and why ecosystems depart from steady state, the
consequences of these departures on ecosystem function and
biodiversity dynamics, including nutrient cycling and invasibility,
and finally, how maximum entropy theory can be used as a tool to
identify these departures.  Remote island archipelagos provide an
opportunity to integrate ecological and evolutionary processes,
advancing our understanding of the regulation of biodiversity through
the lens of theory.  This is particularly true when the component
islands are arranged chronologically, as is found in ``hotspot''
islands that form a geological age gradient representing snapshots of
community assembly through evolutionary time. Such islands provide
simple and discrete systems,of known age and varying area, allowing
them to serve as excellent ``natural laboratories'' for ecological and
evolutionary study in a regional context \citep{simon1987,
  chadwick1999, gillespieClague2009}. Our team has a strong foundation
of research expertise and experience across the islands on microbes
(Brodie), arthropods (Rominger, Gillespie, Gruner and Krehenwinkel),
plants (Chase), ecosystems (Giardina) and theory (Rominger and Chase).

We will characterize the ecological communities, including their
abundance, diversity and network structure, associated with three
critical stages in nutrient cycling: 1) Living plants, the arthropods
they support and the microbes supported by both; 2) Plant and animal
detritus and its associated arthropod and microbial communities; and
3) Soil communities of arthropods and microbes.  In each of these
ecosystem domains we will use the maximum entropy theory of ecology to
characterize departure from statistical steady state.  In order to
understand the mechanistic causes of these departures we will also
evaluate how deviations from METE can be predicted by the ecology and
evolution of the organisms comprising each community, testing the
hypotheses outlined below.  We will enable this line of research by
deliberately sampling plants, arthropods and microbes across multiple
spatial scales, and across gradients of environment (precipitation and
elevation as a surrogate for temperature) and substrate age (as a
surrogate for both biogeochemical change and evolutionary
development).  We will also make use of long term fertilization
experiments \citep{vitousek1997nutrient} to evaluate the orthogonal
roles of evolutionary history versus biogeochemical processes in
driving biodiversity patterns.  Using plants, arthropods and microbes
as discrete test cases, representing a breadth of life history
strategies across the tree of life, we will test hypotheses (outlined
in Box 2) about deviations from statistical steady state based on how
organisms persist, adapt and speciate in their environments.  In order
to understand how communities are likely to change in response to
non-analog, anthropogenically-driven climate regimes and across
spatial scales we will build spatially explicit models that link the
mechanistic drivers (e.g. rapid community or population change, and
evolutionary novelty) of deviation from statistical steady state to
remotely sensed data and detailed ecosystem characterizations taken at
the NEON site in Hawaii, and our complementary sampling locations. Our
project will contribute theoretical constructs for use across NEON
sites and bioinformatic tools to advance the rate and dimensionality
of biodiversity data gathered at these sites.

\section{Proposed Research}

\subsection{Research objectives and hypotheses}

Our proposed objectives and research products are organized in Figure
\ref{fig:research}. We will use maximum entropy theory to identify
deviation from statistical steady state across environmental and
evolutionary gradients, and long-term experiments.  We will place
these deviations in the context of ecological and evolutionary
information to understand the mechanistic causes for deviations from
statistical steady state and its implications for invasion potential.
To forecast these mechanisms and implications into future, non-analog
environments we will model the ecological and evolutionary drivers of
deviations using remotely sensed environmental variables and detailed
field measurements from the NEON site and our complementary sampling
sites.  These models will be spatially explicit and use the framework
of Bayesian hierarchical modeling to incorporate diverse data types.
To permit theory testing and modeling across large scales we will
develop a novel sequencing and bioinformatics approach to generate
massive, multidimensional (i.e. taxonomic and genetic) biodiversity
data.  We will use this combined approach of novel theory testing and
novel data generation to test hypotheses outlined below relating
departures from statistical steady state to feedbacks between
ecological and evolutionary processes.

\subsubsection{Hypotheses}

\begin{itemize}
\item Departures from statistical steady state
  \begin{itemize}
  \item[H1] Deviations from METE are largely predicted by age
    along the chronosequence. These deviations along the
    chronosequence will be driven primarily by two processes
    related to evolutionary assembly of biotas: (H2a) primary
    succession (both by long distance dispersal and speciation) of
    newly formed habitats; and (H2b) adaptive evolution leading to
    unique constraints on assembly not consistent with statistical
    steady state
    \begin{itemize}
    \item[H1a] will be more relevant for generalist taxa,
      especially those that are dispersal limited, on young
      substrates. 
      \begin{itemize}
      \item We predict greatest deviations for communities
        dominated by generalist taxa on young substrates
      \item We predict a positive correlation between deviations
        from METE and measures of spatial turnover, both taxonomic
        and genetic.
      \item We predict a negative correlation between the breadth
        of reconstructed abiotic niches and deviation from METE
      \end{itemize}
    \item[H1b] will be most relevant for specialist taxa once they
      have established intricate evolutionary relationships with
      their coexisting species and environments.
      \begin{itemize}
      \item We predict greatest deviations from METE for
        communities dominated by specialist taxa on old substrates
      \item We predict a positive correlation between network
        specialization and deviation from METE
      \item We predict a negative correlation between phylogenetic
        diversity and deviation from METE
      \end{itemize}
    \item[H1c]  Because niche specialization and dispersal
      limitation both likely result in strong spatial structuring
      of communities, measures of spatial turnover and deviations
      from METE should be correlated across all ages along the
      chronosequence
    \item[H1d] Because rapid population expansion, population
      contraction, limited dispersal and local adaptation all lead
      to low allelic diversity within populations we predict
      genetic diversity to be negatively correlated with deviation
      from METE
    \end{itemize}
  \item[H2] Deviations from METE are not predicted by
    environmental variables after accounting for ecosystem age.
    This includes the prediction that in long term fertilization
    experiments, fertilized communities will conform to the same
    patterns as their unfertalized control communities of the same
    age regardless of underlying nutrient availiblity
  \item[H3] However, with rapidly changing climates we do expect
    environmental predictors of deviations from statistical steady
    state. Specifically, with the creation of novel environments
    and loss of existing environments due to changing climate we
    expect rapid population changes and exacerbated constraints on
    movement due to unique evolutionary adaptations to previously
    stable environments.  Thus we predict novel climatic
    conditions to drive future deviations from METE
  \item[H4] We predict that in disturbed systems the only what for
    statistical steady state to be achieved is through rapid
    assembly of novel ecosystems (i.e. communities dominated by
    highly vagile invasive taxa).  Thus deviations from
    statistical steady state are expected to promote invasion,
    while invasion itself will tend to return systems to
    statistical steady state.
  \end{itemize}
\item Evolution of niches and networks
  \begin{itemize}
  \item[H5] We predict that niches will become more constrained
    across evolutionary time
    \begin{itemize}
    \item[H5a] Reconstructed niches will be smaller for taxa endemic to older islands
    \item[H5b] Spatial turnover will be stronger across gradients on older islands
    \item[H5c] We predict networks will become more specialized
      across evolutionary time
    \end{itemize}
  \end{itemize}
\end{itemize}

\subsection{Significance and Rationale}


Understanding how environmental change will alter the feedback between
ecology and evolution and drive biodiversity out of statistical steady
state is at the core of our proposal.  Using METE to capture
statistical steady state and understand deviations from it promises to
be a powerful diagnostic tool in evaluating ecosystems nearing tipping
points.  Hawaii is an ideal study system to realize this potential due
to its varying chronology (allowing tests of theory in communities of
different stages of evolutionary development) and due to its
replicated environmental gradients across this chronology (see
Fig. \ref{fig:map}. The NEON site at Puu Makaala Natural Area Reserve
on Hawaii Island will provide the core measures needed to quantify the
abiotic environment.  We will replicate these measurements across
gradients of elevation and precipitation, using ground-truthed
remotely sensed measurements to provide both fine grain and
broad-scale environmental data products.

The same ability to generate massive amount of environmental data via
remote sensing does not exist for organismal ecology and evolution.
As part of our Dimensions in Biodiversity grant, PIs Rominger and
Krehenwinkel are developing laboratory and bioinformatic methods to
obtain sequence data, and estimates of abundance and biomass for
thousands to millions of arthropods collected via ecological sampling.
As part of the current proposal this promising new approach will be
developed into an open source lab protocol and software package that
can be distributed across all NEON sites.

Our use of METE as a diagnostic tool has been corroborated in the
Hawaiian system with previous and current work.  PI Rominger, with
co-PIs Gillespie and Gruner as collaborators and co-authors, has shown
that deviations from METE show consistent patterns across the
chronosequences for different arthropod guilds with different life
history characteristics (see Box 2).  This work needs to be extended
to other members of the ecosystem to understand its generality and the
mechanistic drivers of deviations from METE, and thus statistical
steady state, need to be better understood and modeled into the future
to understand how ecosystems will respond to changing climates.

\subsection{Methods}

\subsubsection{Integration with NEON and sampling design across
  environmental and age gradients}

\paragraph{NEON site.}
The goal of NEON is to provide ecological data at multiple spatial and
temporal scales. Our plan is anchored with the Pu'u Maka'ala Natural
Area Reserve on the Mauna Loa volcano on the Big Island of Hawaii
(19.553\textdegree, -155.317\textdegree; Fig. \ref{fig:map}), a Core
Terrestrial site with the launch date planned for 2017. The site
represents montane wet forest with mostly native vegetation dominated
by the endemic tree, {\it Metrosideros polymorpha}
(Myrtaceae). However, up to 95\% of the world’s terrestrial climates
are represented in the greater region of the Hawaiian archipelago
\citep{juvik1998}, and a single site will fail to characterize this
tremendous diversity in climate, habitats and species composition. By
replicating core NEON protocols at carefully selected sites with
orthogonal variation in temperature and precipitation, along a
geological chronosequence representing evolutionary time, the Hawaiian
macrosystem will yield the precision of NEON measurements to test
ecological theory and to predict consequences of future changes in
climate. We aim to combine data to be collected with data from sites
across the Hawaiian Islands, in order to understand regional-scale
ecological processes and how these respond to change over space and
time.


\paragraph{Complementary sites.}
We will collect data in an explicit, nested design that allows
integration with the NEON-generated data, while using data from the
entire terrestrial region of the Hawaiian Islands to provide
information on processes of several groups of organisms across
multiple scales. Data will be gathered across elevation and
precipitation gradients from evolutionarily old, middle aged and young
islands (Kaua'i: 4--5 my; Maui: 1--1.5 my; and Hawai'i: 0.001--0.5
my).  On each island we will establish 6 sites (1 ha in size): 3 along
a windward (i.e. high precipitation) elevation gradient and 3 along a
leeward (i.e. low precipitation) elevation gradient
(Fig. \ref{fig:map}).  Windward sites will be constrained to be within
4000--5000 mm annual precipitation, while leeward sites will be
constrained to be within 1500--2500 mm annual precipitation.  We will
consider an elevation gradient from 900 -- 2500 m elevation.  On
Hawai'i Island we will use the area adjacent to the Pu'u Maka'ala NEON
site as one of these 6 sites.  Each site will consist of 3 replicate
plots to insure thorough coverage of local variation. The sampling
locations and design are given in Figure \ref{fig:map}.


% \begin{figure}[!htb]
%   \centering
%   \includegraphics[scale=1]{figs/fig_dStat(3TP).pdf}
%   \caption{Currently shows three important variables and exisitng
%   sampling sites (gray) as well as NEON site (triangle) and the three
%   regions on Hawaii Island, Maui and Kauai where we will have
%   complementary sites.  Will eventually also show }
%   \label{fig:dStat}
% \end{figure}


\paragraph{Sampling approach and collection of organismal data.}
We will select sites in clearly defined ohia/koa montane, wet and
mesic forest communities. The rationale here is that (i) Ohia
({\it Metrosideros polymorpha}) is the dominant canopy tree in these
forests, forming a nearly continuous layer, with patches of
sub-dominant koa ({\it Acacia koa}) and numerous associated understory
trees, shrubs, herbs, and ferns. This forest type (and the presence of
{\it Metrosideros} in particular) has been used as an important landscape
feature in our ongoing work through the Hawaii Dimensions of
Biodiversity, as it has for a generation of studies on long-term
ecosystem development. This constrains sampling to vegetation and
soils of similar physiognomy and evolutionary history, while allowing
major climatic state factors to vary. (ii) The proposed NEON site is
characterized by this forest type. Finally, (iii) {\it Metrosideros} growth
rate, growth form and chemical composition (all detectable by various
satellite and airborne spectroscopic techniques \citep{asner2006,
  asner2009, asner2012} reflects the coupled but nonlinear effects of
ecosystem age and fertility, which in turn affects the community of
organisms in a given forest stand \citep{crews1995,
  gruner2007}. Differences in plant traits can affect the structure of
an entire food web through a series of direct and indirect effects
\citep{gruner2005, bukovinszky2008}.

Figure \ref{fig:site} details the proposed layout of our sampling
plots. Within each 1-ha site, we will establish three 20-m by 20-m
plots to be selected as representative of forest height mean, maximum,
heterogeneity found in that 1-ha site. Within each 20mx20m plot, we
will establish our replicate plots.  Each plot will be further gridded
into 4 m quadrats (100 in total).  Within each quadrat we will record
all tree species $\geq 1$ cm at breast height.  Within three randomly
selected quadrats we will also sample all herbaceous species.  We will
sample all arthropods within each quadrat using timed beating (24
seconds per quadrat).  Within the same three randomly selected
quadrats we will also extract arthropods using Berlese funnels from
litter and soil samples, gridded to 1 m$^2$ cells (in keeping with the
ground beetles collected at the NEON site).  Arthropods will be
collected into RNAlater to preserve their DNA and RNA as well as the
DNA and RNA of their associated microbes and gut contents.  While NEON
protocols focus on ground beetles (Carabidae), mosquitoes (Diptera:
Culicidae), and ticks (order Ixodida), our study will include all
arthropods because ground beetles constitute an eclectic group of
lineages, most often arboreal and unevenly distributed across the main
islands \citep{Liebherr2000}, and there are no native mosquitoes or
ticks \citep{nishida2002}.

Microbial richness and abundance will also be sampled in a gridded
design.  Within three randomly selected quadrats in each plot we will
take a soil sample 100 cm in surface area (10 cm by 10 cm) and 10 cm
deep.  In the lab this will be divided into a regular 2 cm grid and
each will be sequenced.

In all systems, microbial diversity will focus primarily on the Domain
Bacteria due to its phylogenetic breadth, and metabolic and
respiratory plasticity. Bacterial diversity will be estimated using
molecular tools to sequence 16S rRNA gene biomarkers in multiplex
using a barcoding approach. DNA extraction and 16S rRNA gene
amplification and Illumina sequencing will be carried out according to
Earth Microbiome Project standards
(\url{http://www.earthmicrobiome.org/emp-standard-protocols}). Ancilliary
and meta data collection standards will follow the NEON the soil
microbial data collection and metadata tracking worksheet
(\url{http://goo.gl/nE9zPk}).  Microbial 16S rRNA gene data will be
analyzed according to \citet{shi2015}. Richness will be estimated
using both taxonomic (OTUs) and phylogenetic (Faith's phylogenetic
distance) metrics. Absolute bacterial abundances will be determined
using quantitative PCR as described in \citep{shi2015} while relative
abundances of bacterial taxa will be determined based on the fractions
of sequence reads assigned to each taxon using adjustments for rRNA
gene copy number \citep{kembel2012}.  In order to relate bacterial
taxa to metabolic rate we will use observed relationships between rRNA
copy number, genome size and metabolic rate \citep{delong2010}.



\paragraph{Environmental and biogeochemical data}

\begin{enumerate}
\item Plot-level measurements: In each microbial sampling quadrat we
  will deploy data loggers to record air temperature and moisture
  content.  We will similarly deploy data loggers to record soil
  temperature and moisture.  We will also measure soil physical
  characteristics, pH, total carbon, nitrogen, phosphorous and sulfur.
  We will measure monthly litterfall using litter traps as a surrogate
  for nutrient cycling \citep{austin2000, giardina2004} in addition to
  litter chemistry (pH, total carbon, nitrogen, phosphorous and
  sulfur) \citep{NEON}.
\item Remote Sensing and Measurements of Gases: The NEON site will
  track fluxes of gases, such as carbon dioxide (CO2) and water vapor,
  and collects data about physical and chemical climate conditions,
  such as temperature, barometric pressure and visible light or
  Photosynthetically Active Radiation (PAR). Sensors on the NEON tower
  systems track fluxes of gases (CO2, water vapor) and collects data
  about physical and chemical climate conditions, such as temperature,
  humidity, wind, and the amount of gas that is exchanged between the
  atmosphere and the ecosystem. Towers extend past the top of the
  vegetation canopy at each site to allow sensors mounted at the top
  and along the tower to capture the full profile of atmospheric
  conditions from the top of the vegetation canopy to the
  ground. Automated tower sensors collect data continuously to capture
  patterns and cycles across various time periods, ranging from
  seconds to years. Categories of measurements are physical climate
  (aerosols, precipitation, radiation, and temperature, pressure and
  wind; chemical climate (wet deposition, chemistry, isotopes and
  scalar concentrations); net ecosystem exchange: carbon dioxide (CO2)
  flux, soil CO flux, water vapor and latent heat flux, sensible heat,
  total reactive nitrogen (NO2) and ozone (O3).
\item Airborne Remote Sensing: We will make use of both existing and
  planned airborne remote sensing data which can provide information
  on vegetation composition and land cover and will be used in
  particular to examine the complex mosaic of forest structure and
  composition. The NEON Airborne Observation Platform (AOP) measures
  vegetation biochemical and biophysical properties with spectroscopy,
  vegetation structure and biomass with LiDAR, and produces high
  resolution imagery that can be subject to analyses of land use and
  relative cover \citep{NEON}.
\end{enumerate}



\subsubsection{Modeling evolutionary and environmental drivers of
  assembly}


\paragraph{Maximum entropy theory of ecology across gradients of
  environment and age}

To test our hypotheses relating age, environment and
organism/community traints to deviations from METE we will use the R
package {\tt meteR} \citep[developed by Rominger][]{rominger2016} to
evaluate the goodness of fit of METE for soil microbes, arthropods and
plants at our sampling sites across gradients of precipitation,
elevation and age.  Goodness of fit will be measured as the normalized
log likelihood squared \citep[described in][]{rominger2016}. Using
generalized linear models we will evaluate how the goodness of fit
varies between major groups (microbes, arthropods and plants) and as a
function of the underlying age and environment of each site.

To further explore the relative importance of age as a proxy for
evolution versus biogeochemical environment we will use Vitousek’s
long term fertilization experiments to test whether alleviating
nutrient limitations in old and young plots changes the the way in
which arthropod and microbial communities deviate or conform to METE.


\paragraph{Modeling niches, networks and community phylogenetics
  across space and evolutionary time}

We will develop a Bayesian hierarchical modeling framework to
understand how these drivers of deviations from statistical steady
state response to local and regional environments.  In all models we
will incorporate explanatory environmental variables as spatial
averages with an exponentially decaying distance weighted function.
Each variable will receive maximum weight at the point location of the
specimen and exponentially less weight as distance from the point
location increases.  The exponential rate of decay will be fit as a
free parameter in our Bayesian hierarchical model.

We will use island age as an explanatory variable interacting with
environment to evaluate how the niche occupancy and network position
of each species changes with evolutionary age.  Because we will have
phylogenetic data from metabarcoding for all species we will evaluate
patterns of niche occupancy and network position in a phylogenetic
framework, testing hypotheses of whether closely related taxa overlap
or diverge in niche occupancy, and whether more recently diverged
species tend to be generalists or specialists.

To test whether the niche spaces of taxa change across the
chronosequence we will build probabilistic niche models for all
species of plants and arthropods with sufficient data ($n \geq 15$
points per island).  We will use data sources from our gradient plots,
plots from our Hawaii Dimensions in Biodiversity project, digitized
museum specimens and species occurrence data made available reporting
by the Hawaii Division of Land and Natural Resources.  Because the
nature of these data is variable (abundance and presence-only) we will
use Bayesian hierarchical models to combine them into one analysis
\citep{hsdm}.  We jointly model the niches of all species in this
hierarchical approach.

To test how networks evolve across the chronosequence we will quantify
network structure using four complementary approaches: 1) deviation
from the maximum entropy predictions; 2) classic ecological network
metrics of nestedness and modularity; 3) network dissimilarity; and 4)
network specialization.  We will again take a phylogenetic approach to
evaluate how changes in network position of taxa and changes in
overall structure of networks relates to the phylogenetic distance
between component taxa.

Phylogenetic diversity will itself be modeled as a response to age and
environment using the same Bayesian hierarchical approach as niches
and networks.


\subsubsection{Projecting deviations from statistical steady state
  into the future}

Once we understand the connections between network structure, niche
occupancy, population size change, evolutionary diversification and
deviation from statistical steady state, we can use our models for
niches, networks and phylogenetic diversity to project these drivers
into the future and predict where (at a regional scale) statistical
steady stead will be violated.  Using our understanding of how
statistical steady state contributes to invasibility of a community we
will also be able to model invasion risk across scales and into future
climate scenarios.

\subsubsection{Quantifying evolutionary and macroecological patterns
  using metabarcoding}

Next generation sequencing technology has ushered in a revolution in
evolutionary biology and ecology. This revolution has not passed by
taxonomy and spurred various new studies in the field of molecular
barcoding. The current leap in sequencing throughput allows to
routinely perform barcoding studies on bulk samples and analyzing
whole ecosystems \citep{shokralla2015, gibson2014, taberlet2012}. The
large scale recovery of species richness, food web structure, cryptic
species, identification of juveniles and hidden diversity,
e.g. internal parasitoids, promise unprecedented new insights into
ecosystem function and assembly \citep{krehenwinkel2016,
  shokralla2015, gibson2014, taberlet2012}. While species richness can
be routinely identified by sequencing bulk samples, estimating species
abundance remains challenging \citep{elbrecht2015} and severely limits
the application of metabarcoding to many studies. We are developing
wet lab and bioinformatic methods to overcome this issue and
revolutionize the generation of ecological and genetic data. Our
pipeline consists of three steps (Fig. \ref{fig:metab}):

\begin{enumerate}
\item Extraction and sequencing of pooled community samples
\item Matching the resulting sequences to a reference phylogeny for
  identification
\item Using Bayesian hierarchical models to reconstruct unbiased
  estimates of abundance
\end{enumerate}

Step (1) will be released as an open source lab protocol and steps
(2-3) will be developed into an open source {\tt R} package that allows
users to implement these methods in their study systems.  We propose
that our open source pipeline can be implemented across NEON sites to
generate both taxonomic and phylogenetic data for focal taxa.


% Figure \ref{fig:metab} showing pipeline for generating and analysing
% metabarcoding samples.

Preliminary results from controlled experiments show there is a strong
correlation between amount of DNA and total number of reads; however,
this relationship is variable across taxa
(Fig. \ref{fig:metabData}). A Bayesian model is able to capture this
variability across taxa (Fig. \ref{fig:metabData}) and thus indicates
the success of more general applications of the modeling approach to
field collections.

\paragraph{(1) Extraction and sequencing of pooled community samples.}
We will generate sequence information for mixed arthropod community
samples, collected across precipitation gradients on the Hawaiian
Archipelago. The samples will be roughly pre-sorted taxonomically and
grouped into different body size classes to minimize the confounding
factors of abundance and body size in determining amount of DNA per
taxon. We will use amplicon sequencing of the COI barcoding region
\citep{taberlet2012} which has shown the greatest reliability in
preliminary trials.


\paragraph{(2) Matching the resulting sequences to a reference
  phylogeny for identification.}
In order to resolve the taxonomy of sequences derived from mixed
samples we are developing a library of the barcoding region for
species across the Hawaiian archipelago, such that unknown sequences
can be phylogenetically matched to the reference library.
Sequences not found in the tree of all reference sequences will be
grafted and their status as a unique operational taxonomic unit
assessed using a cuttoff of 3\% divergence (Fig. \ref{fig:metab}).
These bioinformatic steps will be included in the {\tt R} package.

In collaboration with taxonomist and ecologist on Hawaii, we are
currently working on generating the barcode reference library for a
diverse range of several hundred Hawaiian arthropod taxa. These taxa
were sampled across the chronosequence of the Hawaiian Archipelago
(Fig. \ref{fig:map}). DNA is extracted from each taxon and reference
sequence generated for the mitochondrial COI barcoding region. To
achieve a comprehensive sampling of the Hawaiian arthropod diversity,
samples from environmental gradients (e.g. precipitation) will be
included in this reference collection. Such gradients have been shown
to have a profound influence on community composition on Hawaii
\citep{zimmerman2012}.

In order to build a robust phylogenetic backbone for our reference
library, the genomic DNA extracts for all species will be sequenced
using the Illumina HiSeq2500. An assembly of the resulting reads
promises to generate near complete mitochondrial genomes and nuclear
ribosomal clusters of each taxon. To support the Illumina short read
assemblies, we will generate long read information by PacBio
sequencing. The resulting sequence information will allow us to
reconstruct a well resolved community-phylogenetic framework for
ecological hypothesis testing.  These same specimens will also be used
to quantify the microbiomes and feeding habits of hundreds of
arthropod species across our sites (discussed further in section
``Quantifying networks of microbes, arthropods and plants'').

\paragraph{(3) Using Bayesian hierarchical models to reconstruct
  unbiased estimates of abundance}
Bayesian hierarchical models permit inference of key quantities
(e.g. abundance) while accounting for multiple sources of error and
leveraging heterogeneous data types to facilitate inference
\citep{royleDorazio}.  The goal of hierarchically modeling
metabarcoding data is to estimate the abundances of species while
correcting for known biases inherent in amplicon-based sequencing.  We
will account for bias from copy number variation and primer affinity
\citep{elbrecht2015} by directly modeling it, while also
using data on the total number of individuals being sequenced, their
body sizes, and the phylogenetic relationship between their sequences
to constrain the estimates to be more accurate
(Fig. \ref{fig:metabMod}).  Furthermore, information from controlled
experiments (for example making mock communities of known composition
and sequencing those) can be used to constrain prior distributions and
obtain even more accurate abundance estimates.


\subsection{Quantifying networks of microbes, arthropods and plants}

Using the specimens reserved from metabarcoding (i.e. those used to
build the reference library and phylogenetic backbone) we will
sequence the microbial associates of each species and their gut
contents, for herbivorous arthropods.  These sequences will allow us
to reconstruct the networks between arthropods and their microbial
associates as well as herbivorous arthropods and their plant hosts.
We will additionally reconstruct microbial networks based on
covariance between prevalence of microbial taxa in samples using
established approaches \citep{kurtz2015}.




%% suppresses bibliography, then make tex file to compile .bbl in separate doc
\bibliographystyle{unsrtnat}
\setbox0\vbox{\bibliography{../macrosystems.bib}}


\end{document}



